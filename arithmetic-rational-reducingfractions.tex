\documentclass[20150903-160354-rs2.2-MarksMathNotebook.tex]{subfiles}

\begin{document}
%-=-=-=-=-=-=-=-=-=-=-=-=-=-=-=-=-=-=-=-=-=-=-=-=
%
%	CHAPTER
%
%-=-=-=-=-=-=-=-=-=-=-=-=-=-=-=-=-=-=-=-=-=-=-=-=

\chapter{Reducing Fractions}


%-=-=-=-=-=-=-=-=-=-=-=-=-=-=-=-=-=-=-=-=-=-=-=-=
%	SECTION:
%-=-=-=-=-=-=-=-=-=-=-=-=-=-=-=-=-=-=-=-=-=-=-=-=

\section{Arithmetic Expressions}\index{Fractions!Reducing Fractions}


%-=-=-= DEFINITION
\begin{definition}[Reduced Fraction]\index{Reduced Fraction}
A fraction of the form $\frac{m}{n}$, where $n \ne 0$, is said to be in it's reduced form if the $\gcd(m,n)=1$
\end{definition}

\begin{figure}[!ht]
\begin{center}
\begin{tikzpicture}[scale=1, node distance=2cm, text width=6em,text centered, auto]
    % Place nodes
\node [bluerbox, text width=9em] (numerator) {factor numerator};
\node [below of =numerator](empty){};
\node [bluerbox, text width=9em, below of = empty] (denominator) {factor denominator};
\node [redrbox, right of = empty] (cancel) {cancel MId};

\path [line] (numerator) -- (denominator);
\path [line] (numerator) edge [out=0, in=90] (cancel);
\path [line] (denominator) edge [out=0, in=270] (cancel);
\end{tikzpicture}
\end{center}
\caption{Reducing Fractions Workflow}
\end{figure}

\begin{figure}[!ht]
\begin{center}
\begin{tikzpicture}[scale=1, node distance=2cm, text width=6em,text centered, auto]
    % Place nodes
\node [bluerbox, text width=9em] (gcd) {$\gcd{n,d}$};
\node [redrbox, below right of = gcd] (reduced) {Reduced Fraction};
\node [bluerbox, text width=9em, below left of = gcd](numerator) {$\frac{n}{\gcd(n,d)}$};
\node [bluerbox, text width=9em, below of = numerator](numerator) {$\frac{d}{\gcd(n,d)}$};

\path [line] (gcd) edge [out=270, in=90] (numerator);
\path [line] (gcd) edge [out=270, in=90] (reduced);
\path [line] (denominator) edge [out=0, in=270] (reduced);
\end{tikzpicture}
\end{center}
\caption{Using the $\gcd(n,d)$ to reduce the fraction $\frac{n}{d}$ Workflow}
\end{figure}


\end{document}

