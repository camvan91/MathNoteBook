\documentclass[20150903-160354-rs2.2-MarksMathNotebook.tex]{subfiles}

\begin{document}
%-=-=-=-=-=-=-=-=-=-=-=-=-=-=-=-=-=-=-=-=-=-=-=-=
%
%	CHAPTER
%
%-=-=-=-=-=-=-=-=-=-=-=-=-=-=-=-=-=-=-=-=-=-=-=-=

\chapter{Equations of Tangent \& Secant Lines}

%-=-=-=-=-=-=-=-=-=-=-=-=-=-=-=-=-=-=-=-=-=-=-=-=
%	SECTION: Essential Questions
%-=-=-=-=-=-=-=-=-=-=-=-=-=-=-=-=-=-=-=-=-=-=-=-=

\section{Essential Questions}\label{Essential Questions}

\begin{essentialq}\hfill \\

\begin{enumerate}
	\item How do we find the equation of the tangent line of a given function at the point $P(a,b)$?
	\item How do we find the equation of the tangent line of a given function at $x=a$?
\end{enumerate}

\end{essentialq}

%-=-=-=-=-=-=-=-=-=-=-=-=-=-=-=-=-=-=-=-=-=-=-=-=
%	SECTION: Finding the Equation of the Tangent Line 
%-=-=-=-=-=-=-=-=-=-=-=-=-=-=-=-=-=-=-=-=-=-=-=-=

\section{Finding the Equation of the Tangent Line }\label{Finding the Equation of the Tangent Line }
\begin{figure}[!ht]
\begin{center}
\begin{tikzpicture}[scale=1, node distance=2cm, text width=3em,text centered, auto]
    % Place nodes
\node [bluerbox] (abscissa) {$a$};
\node [bluerbox, right of = abscissa] (ordinate) {$b$};
\node [greenrbox, below of = abscissa] (function) {$f(x)$};
\node [circlenode, right of = function] (join) {$b$};
\node [purplerbox, below of = function] (derivative) {$f'(x)$};
\node [purplerbox, right of = derivative] (derivativea) {$f'(a)$};
\node [orangerbox, right of = join, node distance=4cm, text width=8em] (intercept) {$k=-f'(a)a+b$};
\node [redrbox, below of= intercept, text width=8em] (equation) {$y=f'(a)x+k$};

\path [line] (abscissa) edge [out= 180, in= 180] (function);
\path [line] (function) -- (join);
\path [line] (ordinate) edge [out=270, in=90] (join);
\path [line] (join) -- (intercept);
\path [line] (abscissa) edge [out= 180, in= 180] (derivative);
\path [line] (derivative) edge [out=0, in=180] (derivativea);
\path [line] (derivativea) edge [out=0, in=180] (intercept);
\path [line] (abscissa) edge [out=90, in=90] (intercept);
\path [line] (intercept) edge [out=0, in=0] (equation);
\path [line] (derivative) edge [out=0, in=180] (derivativea);
\path [line] (derivativea) edge [out=0, in=180] (equation);
\path [dashline] (function)--(derivative);
\end{tikzpicture}
\end{center}
\caption{Finding the Equation of a Tangent Line Workflow}
\end{figure}

For a given function, $\cGreen{f(x)}$, our goal is to find the equation of the tangent line at the point $P(\cBlue{a}, \cBlue{b})$, which can be expressed in slope-intercept form as:
\[
y=\cPurple{f'(\cBlue{a})}x+\cOrange{k}
\] 
Where $\cPurple{f'(\cBlue{a})}$ is the value of the derivative, slope, at $x=\cBlue{a}$ and $\cOrange{k}$ is the $y$-intercept.  We therefore need to:

\begin{enumerate}
	\item Find the derivative of the function $\cGreen{f(x)}$: $\cPurple{f'(x)}$
	\item Find the value of the derivative at $x=\cBlue{a}$: $\cPurple{f'(\cBlue{a})}$
	\item Find the value of the $y$-intercept: $\cOrange{k}=-\cPurple{f'(\cBlue{a})}+\cBlue{b}$:
	\item If the ordinate $\cBlue{b}$ is not explicitly given, then find $\cGreen{f(\cBlue{a})}=\cBlue{b}$ 
\end{enumerate}

%-=-=-= EXAMPLE
\begin{example}[id:20151011-154209] \label{20151011-154209}\index{Example!20151011-154209} \hfill \\
Find the equation of the line tangent to the curve of the function $f(x)=2x^2+3x+7$ at the point $P(2, 21)$.

\soln

\solnsteps

Find the derivative of $f(x)$
\begin{align*}
f'(x) &= 4x+3 \text{\, goto \,} \, \ref{20151011-195002}
\end{align*}

Evaluate the derivative at $x=2$
\begin{align*}
f'(2) & = 4\farg{2}+3 && \text{SPE} \eqref{eq:spe} \\
f'(2) &= 8+3 && \text{OOM} \eqref{eq:oom} \\
f'(2) &= 11 && \text{OOA} \eqref{eq:ooa} 
\end{align*}

Find the $y$-intercept, $k$, of the equation of the tangent line.

\begin{align*}
y &= f'(x)x + k\\
\farg{21} & = \farg{11}\farg{2} + k && \text{SPE} \eqref{eq:spe} \\
21 & = 22 + k && \text{OOM} \eqref{eq:oom} \\
\neg 22 + \left[\farg{21} \right] &= \neg 22 + \left[\farg{22} \right] + k && \text{SPE+AI}\\ 
\neg 22 + 21 & = (\neg 22 + 22) + k && \text{APA} \eqref{eq:apa1} \\
\neg 1 &= 0 + k && \text{OOA} \eqref{eq:ooa} \\
\neg 1 &= k && \text{AId} \eqref{eq:aid1} \\
-1 &= k && \text{ONeg} \eqref{eq:oneg2} \\
k &= -1 && \text{SyPE} \eqref{eq:sype} 
\end{align*}

The equation of the tangent line is
\begin{align*}
y & = f'(2)x+k \\
y &= 11x - 1 && \text{SPE} \eqref{eq:spe} 
\end{align*}

\end{example}

\end{document}

