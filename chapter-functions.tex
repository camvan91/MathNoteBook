%!TEX root = /Users/markholson/Dropbox/+Projects/LatexFiles/Book/main.tex
%-=-=-=-=-=-=-=-=-=-=-=-=-=-=-=-=-=-=-=-=-=-=-=-=
%
%	CHAPTER 
%
%-=-=-=-=-=-=-=-=-=-=-=-=-=-=-=-=-=-=-=-=-=-=-=-=

\chapterimage{chapter_head_2.pdf} % Chapter heading image

\chapter{Functions}

%-=-=-=-=-=-=-=-=-=-=-=-=-=-=-=-=-=-=-=-=-=-=-=-=
%	SECTION: Evaluating Functions
%-=-=-=-=-=-=-=-=-=-=-=-=-=-=-=-=-=-=-=-=-=-=-=-=

\section{Evaluating Functions}\label{Evaluating Functions}

\begin{example}[id:20141106-083703]\label{20141106-083703}\index{Example!20141106-083703} \hfill \\
Given $f(x)= \dfrac{1}{12}x^3-x^2+4x$, evaluate $f(3)-f(2)$.

\soln

\solnsteps
\begin{align*}
f(3)-f(2) & = \left(\dfrac{\farg{3}^3}{12} - \farg{3}^2 + 4\farg{3} \right) - \left(\dfrac{\farg{2}^3}{12} - \farg{2}^2 + 4\farg{2} \right) \\
&= \left(\dfrac{(3)^3}{12} - (3)^2 + 4(3) \right) - 1 \left(\dfrac{(2)^3}{12} - (2)^2 + 4(2) \right) && \text{MId} \eqref{eq:mid1} \\
&= \left(\dfrac{(3)^3}{12} + \neg  (3)^2 + 4(3)\right) + \neg 1 \left(\dfrac{(2)^3}{12} + \neg (2)^2 + 4(2) \right) && \text{DOS} \eqref{eq:dos1} \\
&= \left(\dfrac{27}{12} + \neg 9 + 4(3) \right)+ \neg 1 \left(\dfrac{8}{12} + \neg 4 + 4(2) \right) && \text{OOE} \eqref{eq:ooe} \\
&= \left(\dfrac{27}{12} + \neg 9 + 12 \right) + \neg 1 \left(\dfrac{8}{12} + \neg 4 + 8 \right) && \text{OOM} \eqref{eq:oom} \\
&= \left(\dfrac{27}{12} + 3 \right) + \neg 1 \left(\dfrac{8}{12} + 4 \right) && \text{OOA} \eqref{eq:ooa} \\
&= \left(\dfrac{27+36}{12} \right) + \neg 1 \left(\dfrac{8+48}{12}\right) && \text{FOOA} \eqref{eq:fooa1} \\
&= \left(\dfrac{63}{12} \right) + \neg 1 \left(\dfrac{56}{12}\right) && \text{OOA} \eqref{eq:ooa} \\
&= \left(\dfrac{63}{12} \right) + \neg \dfrac{56}{12} && \text{OOM} \eqref{eq:oom} \\
&= \dfrac{7}{12} && \text{OOA} \eqref{eq:ooa} 
\end{align*}

\qdepend 

\qdependlist

example \ref{20141105-144223}-20141105-144223


\end{example}

%-=-=-=-=-=-=-=-=-=-=-=-=-=-=-=-=-=-=-=-=-=-=-=-=
%	SECTION: Rational Functions
%-=-=-=-=-=-=-=-=-=-=-=-=-=-=-=-=-=-=-=-=-=-=-=-=

\section{Quadratic Functions}\label{Quadratic Functions}

\subsection{Completing the Square}

\begin{definition}[Completing The Square]
Completing the square is the process used to convert a quadratic polynomial function \(f(x)=ax^2+bx+c\) to the form

\[ f(x)=a \left(x+\dfrac{b}{2a} \right)^2 + \left(c-\dfrac{b^2}{4a} \right) \]

We can simplify this form by defining \(B = \dfrac{b}{2a} \) and \(C=c-\dfrac{b^2}{4a} \), which gives us 

\[ f(x)=a \left(x+B \right) + C \] 

\hfill \cite{mathworld:completethesquare}
\end{definition}

\begin{align*}
f(x) 	&= ax^2+bx+c \\
		&= a \left[x^2+ \dfrac{b}{a}x + \dfrac{c}{a} \right] && \text{DPF} \eqref{eq:dpf2} \\
		&= a \left[x^2 + \dfrac{b}{a}x + \cRed{k} + \cRed{\neg k} + \dfrac{c}{a} \right] && \text{AId} \eqref{eq:aid2} \\
		&= a \left[ \left(x^2 + \dfrac{b}{a}x + \cRed{k} \right)+ \left(\cRed{-k}+ \dfrac{c}{a}  \right) \right] && \text{APA} \eqref{eq:apa2} \\
\end{align*}

\begin{figure}[h!]
\centering
\begin{tikzpicture}[scale=1, auto]

% Place nodes

\node[firstterm](11){$x$}; \node[factoradd,right=of 11](plus1){}; \node[secondterm, right=of plus1](12){$\frac{b}{2a}$};
\node[firstterm, below=of 11](21){$x$}; \node[factoradd,right=of 21](plus2){}; \node[secondterm, right=of plus2](22){$\frac{b}{2a}$};

\node[multiply, below=of 21](31){$x^2$};
\node[multiply, below=of 22](32){$\cRed{k}=\frac{b^2}{4a^2}$};

\node[multiply, right=of 12](13){$\frac{b}{2a}x$};
\node[multiply, right=of 22](23){$\frac{b}{2a}x$};

\node[add, below=of 23](33){$\frac{b}{2}x$};

\path [line](11) edge[bend right=30]node[color=black, midway, left]{$\times$}(21);
\path [line](12) edge[bend left=30]node[color=black,, midway, right]{$\times$}(22);
\path [line](21)--(31);

\path [line](21) edge[bend left=30]node[color=black, pos=0.4, below]{$\times$}(12);
\path [line](11) edge[bend right=30](22);
\path [line](22)--(32);

\path [line](12)--(13);
\path [line](22)--(23);

\path [line](13)--node[color=black, midway, right]{$+$}(23);
\path [line](23)--(33);

\end{tikzpicture}
\caption{Factoring Organizer used to find the value of $\cRed{k}$}
\end{figure}

\begin{align*}
f(x) 	&=  a \left[ \left(x^2 + \dfrac{b}{a}x + \cRed{\dfrac{b^2}{4a^2}} \right)+ \left(\cRed{\dfrac{-b^2}{4a^2}}+ \dfrac{c}{a}  \right) \right]\\
		&= a \left[ \left(x+ \dfrac{b}{2a} \right)^2 + \cRed{\dfrac{-b^2}{4a^2}}+ \dfrac{c}{a} \right]  && \text{DPF} \eqref{eq:dpf2} \\
		&= a \left[  \left(x+ \dfrac{b}{2a} \right)^2 + \dfrac{-b^2}{4a^2}+ \dfrac{4ac}{4a^2} \right] \\ % TODO Common Denominator Step
		&= a \left[  \left(x+ \dfrac{b}{2a} \right)^2 + \dfrac{4ac-b^2}{4a^2} \right]  && \text{OOA} \eqref{eq:ooa} \\
		&= a \left(x+ \dfrac{b}{2a} \right)^2 + \dfrac{4ac-b^2}{4a} && \text{DPE} \eqref{eq:dpe1} \\ 
		&= a \left(x+ \dfrac{b}{2a} \right)^2 + \left(c-\dfrac{b^2}{4a} \right) % TODO Reduce fraction step 
\end{align*}

%-=-=-=-=-=-=-=-=-=-=-=-=-=-=-=-=-=-=-=-=-=-=-=-=
%	SECTION: Rational Functions
%-=-=-=-=-=-=-=-=-=-=-=-=-=-=-=-=-=-=-=-=-=-=-=-=

\section{Rational Functions}\label{Rational Functions}

%-=-=-= DEFINITION
\begin{definition}[Asympote]\index{Asympote}

An asymptote is a line or curve that apporoaches a given curve arbitrarily close.\cite{mathworld:asymptote} \\

A vertical asymptote is a vertical line $x_{va}=c$, that is used to visualize the values of $x$ for which the function is not defined. 

\end{definition}

\begin{figure}[ht]
\begin{center}
	\begin{tikzpicture}
	\begin{axis}[
            domain=-6:2,
            ymax=4,
            ymin=-4,
            samples=100,
            axis lines =middle, xlabel=$x$, ylabel=$y$,
            every axis y label/.style={at=(current axis.above origin),anchor=south},
            every axis x label/.style={at=(current axis.right of origin),anchor=west},
			restrict y to domain=-20:20
          ]
          \addplot [dashed, stockholmPink, smooth] plot coordinates {(-2,4) (-2,-4)}; %% {.451};

          \addplot [very thick, stockholmBlue, smooth] {1/(x+2)};

          \node at (axis cs:-5.7,3.8) [anchor=west] {\color{stockholmPink}Vertical Asymptote};  

        \end{axis}
\end{tikzpicture}

\end{center}
\caption{Vertical Asymptote}
\label{figure:rectangularhyperbola}
\end{figure}

%-=-=-= EXAMPLE
\begin{example}[id:20141111-190212] \label{20141111-190212}\index{Example!20141111-190212} \hfill \\

Find the vertical asymptote of the function $R(x)=\dfrac{7}{x+8}$

\soln

\solnsteps

We are interested in the values of $x$ for which the denominator of $R(x)$ has a value of zero.

\begin{align*}
x_{va}+8 &= 0 \\
x_{va} &=-8  &&\text{solving} 
\end{align*}

\begin{center}
	\begin{tikzpicture}
	\begin{axis}[
            domain=-20:10,
            ymax=10,
            ymin=-10,
            samples=100,
            axis lines =middle, xlabel=$x$, ylabel=$y$,
            every axis y label/.style={at=(current axis.above origin),anchor=south},
            every axis x label/.style={at=(current axis.right of origin),anchor=west},
			restrict y to domain=-20:20
          ]
          \addplot [dashed, stockholmPink, smooth] plot coordinates {(-8,10) (-8,-10)}; %% {.451};

          \addplot [very thick, stockholmBlue, smooth] {7/(x+8)};

          \node at (axis cs:-17.7,5.8) [anchor=west] {\color{stockholmPink}$x_{va}=-8$};  

        \end{axis}
\end{tikzpicture}

\end{center}
\end{example}

%-=-=-= EXAMPLE
\begin{example}[id:20141111-192213] \label{20141111-192213}\index{Example!20141111-192213} \hfill \\

Find the vertical asymptote of the function $R(x)=\dfrac{x+4}{2x+5}$

\soln

\solnsteps

We are interested in the values of $x$ for which the denominator of $R(x)$ has a value of zero.

\begin{align*}
2x_{va}+5 &= 0 \\
x_{va} &= -\dfrac{5}{2}  &&\text{solving \ref{20141111-215726}} \\
\end{align*}

\begin{center}
	\begin{tikzpicture}
	\begin{axis}[
            domain=-6:2,
            ymax=5,
            ymin=-5,
            samples=100,
            axis lines =middle, xlabel=$x$, ylabel=$y$,
            every axis y label/.style={at=(current axis.above origin),anchor=south},
            every axis x label/.style={at=(current axis.right of origin),anchor=west},
			restrict y to domain=-5:5
          ]
          \addplot [dashed, stockholmPink, smooth] plot coordinates {(-5/2,4) (-5/2,-4)}; %% {.451};

          \addplot [very thick, stockholmBlue, smooth] {(x+4)/(2*x+5)};

          \node at (axis cs:-4.7,2.8) [anchor=west] {\color{stockholmPink}$x_{va}=-\frac{5}{2}$};  

        \end{axis}
\end{tikzpicture}

\end{center}
\end{example}







