\documentclass[20150903-160354-rs2.2-MarksMathNotebook.tex]{subfiles}

\begin{document}
%-=-=-=-=-=-=-=-=-=-=-=-=-=-=-=-=-=-=-=-=-=-=-=-=
%
%	CHAPTER
%
%-=-=-=-=-=-=-=-=-=-=-=-=-=-=-=-=-=-=-=-=-=-=-=-=

\chapter{Properties}


%-=-=-=-=-=-=-=-=-=-=-=-=-=-=-=-=-=-=-=-=-=-=-=-=
%	SECTION:
%-=-=-=-=-=-=-=-=-=-=-=-=-=-=-=-=-=-=-=-=-=-=-=-=

\section{Properties of Addition}\index{Properties of Addition}


%-=-=-= DEFINITION
\begin{definition}[Natural Numbers]\index{Number System! Natural Numbers}

\[
\mathbb{N}=\set{0, 1, 2, 3 \ldots}
\]

\end{definition}

\begin{remark}
It is not uncommon for zero to be excluded from the natural numbers.  In fact, some exclude zero from the natural numbers and then describe the set of natural numbers that include zero the whole numbers. \\

\[
\mathbb{W}=\set{0, 1, 2, 3, \ldots}
\]

For the purposes of these notes, zero will be included within the set of natural numbers.
\end{remark}

%-=-=-= DEFINITION
\begin{definition}[Operation of Addition (OOA)]\index{Operation!Operation of Addition}
\begin{align}
\underbrace{\underbrace{a}_{\text{Augend}}+\underbrace{b}_{\text{Addend}}}_{\text{Sum}} \label{eq:ooa}
\end{align}

More generally,


\begin{align}
\underbrace{\underbrace{a}_{\text{Summand}}+\underbrace{b}_{\text{Summand}}}_{\text{Sum}} \label{eq:ooag}
\end{align}
\end{definition}

%-=-=-= DEFINITION
\begin{definition}[Operation of Multiplication (OOM)] \index{Operation!Operation of Multiplication}
\begin{align}
\underbrace{\underbrace{a}_{\text{Multiplicand}} \times \underbrace{b}_{\text{Multiplier}}}_{\text{Product}} \label{eq:oom}
\end{align}

More generally,

\begin{align}
\underbrace{\underbrace{a}_{\text{Factor}} \times \underbrace{b}_{\text{Factor}}}_{\text{Product}} \label{eq:oomg}
\end{align}

\end{definition}



\begin{definition}[Integers]\index{Number System! Integers}

\[
\mathbb{Z}=\set{\ldots, -3, -2, -1, 0, 1, 2, 3, \ldots}
\]

\end{definition}

%-=-=-= DEFINITION
\begin{definition}[Positive Integers]\index{Positive Integers}

\[
\mathbb{Z}^{+}=\set{1, 2, 3, \ldots}
\]
\end{definition}


%-=-=-= DEFINITION
\begin{definition}[Greatest Common Divisor]\index{Greatest Common Divisor}

Suppose that $m$ and $n$ are positive integers.  The greatest common divisor is the largest divisor (factor) common to both $m$ and $n$.

\end{definition}

%-=-=-= DEFINITION
\begin{definition}[Relatively Prime]\index{Relatively Prime}

Two integers $m$ and $n$ are relatively prime to each other, $m \perp n$, if they share no common positive integer divisors (factors) except 1.

\[
m \perp n \, \text{if} \, \gcd(m, n)=1.
\]

\end{definition}

\begin{definition}[Rational Numbers]\index{Number System! Rational Numbers}

\[
\mathbb{Q}=\set{m/n \suchthat m,n \in \mathbb{Z}, n \ne 0}
\]
\end{definition}

%-=-=-= DEFINITION
\begin{definition}[Proper Fraction]\index{Proper Fraction}

Given $m<n$, then the fraction $m/n$ is called \alert{proper}.

\end{definition}

%-=-=-= DEFINITION
\begin{definition}[Improper Faction]\index{Improper Faction}

Given $m>n$, then the fraction $m/n$ is called \alert{improper}.

\end{definition}

%-=-=-= DEFINITION
\begin{definition}[Common Denominator (CD)]\index{Common Denominator}
\begin{subequations}
\begin{align}
\dfrac{a}{b} + \dfrac{c}{b} &= \dfrac{a+c}{b} \label{eq:cd1} \\
\dfrac{a+c}{b}&= \dfrac{a}{b} + \dfrac{c}{b} \label{eq:cd2}
\end{align}
\end{subequations}
\end{definition}

%-=-=-= RULE
\begin{arule}[Fraction Operation of Addition (FOOA)]\index{Fraction Operation of Addition}
\begin{subequations}
\begin{align}
\dfrac{a}{b} + \dfrac{c}{d} &= \dfrac{ad+bc}{bd} \label{eq:fooa1} \\
\dfrac{ad+bc}{bd} &= \dfrac{a}{b} + \dfrac{c}{d} \label{eq:fooa2}
\end{align}
\end{subequations}
\end{arule}

\end{document}

