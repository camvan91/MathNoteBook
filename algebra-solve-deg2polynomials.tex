\documentclass[20150903-160354-rs2.2-MarksMathNotebook.tex]{subfiles}

\begin{document}
%-=-=-=-=-=-=-=-=-=-=-=-=-=-=-=-=-=-=-=-=-=-=-=-=
%
%	CHAPTER
%
%-=-=-=-=-=-=-=-=-=-=-=-=-=-=-=-=-=-=-=-=-=-=-=-=

\chapter{Solving Quadratic Equations}

%-=-=-=-=-=-=-=-=-=-=-=-=-=-=-=-=-=-=-=-=-=-=-=-=
%	SECTION:
%-=-=-=-=-=-=-=-=-=-=-=-=-=-=-=-=-=-=-=-=-=-=-=-=
\section{Power Inverse}\index{Solving Quadratic Equations! Power Inverse}
%-=-=-= EXAMPLE
\begin{example}[id:20141107-131748] \label{20141107-131748} \index{Example!20141107-131748} \hfill \\

Solve the equation $2-x^2=0$ for $x$

\soln

\solnsteps
\begin{align*}
2-1x^2 &= 0 && \text{MId} \eqref{eq:mid1} \\
2+\neg 1 x^2 &= 0 && \text{DOS} \eqref{eq:dos1} \\
\left[\alert{2+\neg 1x^2} \right] + 1x^2 &= \left[\alert{0}\right]+ 1x^2 && \text{SPE} \eqref{eq:spe} + \text{AI} \eqref{eq:ai} \\
2 + \left(\neg 1x^2 + 1x^2 \right) &= 0 + 1x^2 && \text{APA} \eqref{eq:apa1} \\
2 + 0 &= 0 + 1x^2 && \text{OOA} \eqref{eq:ooa} \\
2 &= 1x^2 && \text{AId} \eqref{eq:aid1} \\
2 &= x^2 && \text{MId} \eqref{eq:mid2} \\
\alert{\pm} \left[\alert{2} \right]^{\frac{1}{2}} &= \left[\alert{x^2}\right]^{\frac{1}{2}} \\ % TODO Power Inverse
\pm 2^{\frac{1}{2}} &= x && \text{PoPo} \eqref{eq:popo1} \\
\pm \sqrt{2} &= x && \text{PoTR} \eqref{eq:potr} \\
x &= \pm \sqrt{2} && \text{SyPE} \eqref{eq:sype}
\end{align*}
\end{example}

%-=-=-= EXAMPLE
\begin{example}[id:20151012-192313] \label{20151012-192313}\index{Example!20151012-192313} \hfill \\

Solve the equation $6x^2 - 6x - 12 = 0$

\soln

\solnsteps
\begin{align*}
6x^2+\neg 6x + \neg 12 &= 0 && \text{DOS} \eqref{eq:dos1} \\
6(1x^2 \neg 1x + \neg 2) &= 0 && \text{DPF} \eqref{eq:dpf2} \\
\end{align*}

Using the factor orgnanizer,

\begin{tikzpicture}[scale=1, auto]

% Place nodes

\node[firstterm](11){$x$}; \node[factoradd,right=of 11](plus1){$+$}; \node[secondterm, right=of plus1](12){$\neg 2$};
\node[firstterm, below=of 11](21){$x$}; \node[factoradd,right=of 21](plus2){$+$}; \node[secondterm, right=of plus2](22){$1$};

\node[multiply, below=of 21](31){$x^2$};
\node[multiply, below=of 22](32){$\neg 2$};

\node[multiply, right=of 12](13){$\neg 2 x$};
\node[multiply, right=of 22](23){$1 x$};

\node[add, below=of 23](33){$\neg 2x$};

\path [line](11) edge[bend right=30]node[color=black, midway, left]{$\times$}(21);
\path [line](12) edge[bend left=30]node[color=black,, midway, right]{$\times$}(22);
\path [line](21)--(31);

\path [line](21) edge[bend left=30]node[color=black, pos=0.4, below]{$\times$}(12);
\path [line](11) edge[bend right=30](22);
\path [line](22)--(32);

\path [line](12)--(13);
\path [line](22)--(23);

\path [line](13)--node[color=black, midway, right]{$+$}(23);
\path [line](23)--(33);

\end{tikzpicture}

Solving the linear equations using \text{ZPr} \eqref{eq:zpr} \\ \\
\begin{align*}
x_1+\neg 2& = 0  &&\text{\alert{Case I}} \\
\farg{x_1 +\neg 2}  + 2 &= \farg{0} + \neg 2 && \text{AI} \eqref{eq:ai},\text{SPE} \eqref{eq:spe}  \\
x_1+(\neg 2 + 2) &= 0 + \neg 2 && \text{APA} \eqref{eq:apa1} \\
x_1+0 &= \neg 2 && \text{OOA} \eqref{eq:ooa} \\
x_1&= \neg 2 && \text{AId} \eqref{eq:aid2} \\
x_1 &= -2 && \text{ONeg} \eqref{eq:oneg2} 
\end{align*}
\vspace{-0.5cm}
\begin{align*}
x_2+1 &= 0  &&\text{\alert{Case II}} \\
\farg{x_2+1} + \neg 1 &= \farg{0} + \neg 1 && \text{AI} \eqref{eq:ai},\text{SPE} \eqref{eq:spe}  \\
x_2 + (1 + \neg 1) &= 0+ \neg 1 && \text{APA} \eqref{eq:apa1} \\
x_2 + 0 &= \neg 2 && \text{OOA} \eqref{eq:ooa} \\
x_2 &= \neg 2 && \text{AId} \eqref{eq:aid2} \\
x_2 &= -2 && \text{ONeg} \eqref{eq:oneg2} 
\end{align*}

\qdepend

\qdependlist
example \ref{20151012-190708}-20151012-190708

\end{example}

%-=-=-=-=-=-=-=-=-=-=-=-=-=-=-=-=-=-=-=-=-=-=-=-=
%	SECTION:
%-=-=-=-=-=-=-=-=-=-=-=-=-=-=-=-=-=-=-=-=-=-=-=-=
\section{Completing The Square}\index{Solving Quadratic Equations! Completing the Square}
Completing the square is an algebraic algorithm used to find the solutions of quadratic equations of the form, $ax^2+bx+c=0$.   Essentially, we want to manipulate this equation such that $x=\text{some values}$.  To gain some understanding of how this algorithm works, we will consider each step individually.  Let's begin with a quadratic equation in the general form: $ax^2+bx+c=0$.\\

Since we are trying to manipulate the equation  $ax^2+bx+c=0$ such that $x=\text{some value}$, we first want the coefficient factor $a$ to be equal to 1.  This is done by multiplying both expressions by the multiplicative inverse of $a$ (step 1) followed by simplifying both expressions.


\begin{align*}
\dfrac{1}{a} \farg{ax^2 + bx +c} & = \dfrac{1}{a} \farg{0} && \text{SPE} \eqref{eq:spe} + \text{MI} \eqref{eq:mi1} \\
\dfrac{1}{a} \cdot ax^2 + \dfrac{1}{a} \cdot bx + \dfrac{1}{a} \cdot c  & = \dfrac{1}{a} \left[ 0\right] && \text{DPE} \eqref{eq:dpe1} \\
\dfrac{1}{a} \cdot a \cdot x^2 + \dfrac{1}{a} \cdot b \cdot x + \dfrac{1}{a} \cdot c  & = \dfrac{1}{a} \left[ 0\right] && \text{JTC} \eqref{eq:jtc} \\
x^2 + \dfrac{b}{a} \cdot x + \dfrac{c}{a} & = 0 && \text{OOM} \eqref{eq:oom} \\
x^2 +\dfrac{b}{a}x + \dfrac{c}{a}  & = 0 && \text{CTJ} \eqref{eq:ctj}
\end{align*}

We now have three terms in the left hand expression where the first two terms have at least one variable factor, $x$.  The problem is that the third term is a constant and we want $x=\text{some value}$.  This text step is to get rid of the $\dfrac{c}{a}$ term, which can be done by using the additive inverse followed by simplifying both expressions.\\

\begin{align*}
\left[x^2 +\dfrac{b}{a}x + \dfrac{c}{a} \right]  + \neg \dfrac{c}{a}  & = \left[ 0 \right]   + \neg \dfrac{c}{a} && && \text{SPE} \eqref{eq:spe} + \text{AI} \eqref{eq:ai} \\
x^2 +\dfrac{b}{a}x + \dfrac{c}{a} + \neg \dfrac{c}{a}  & = 0  + \neg \dfrac{c}{a} && && \text{APA} \eqref{eq:apa1} \\
x^2 +\dfrac{b}{a}x  & = \neg \dfrac{c}{a} && \text{OOA} \eqref{eq:ooa} \\
\end{align*}



The next step is called completing the square.  The idea is to add a \textit{NeW} constant, $\color{sthlmRed}k$, to the left-hand expression, $x^2 +\dfrac{b}{a}x+\color{sthlmRed}k$,  such that the quadratic expression can then be factored as two identical factors, $({\color{sthlmBlue}x}+{\color{sthlmRed}m})({\color{sthlmBlue}x}+{\color{sthlmRed}m}) = ({\color{sthlmBlue}x}+{\color{sthlmRed}m})^2$, where  ${{\color{sthlmRed}k}=\color{sthlmRed}m} \cdot {\color{sthlmRed}m}$.  Notice that since we are adding a constant term, $\color{sthlmRed}k$, to the left-hand expression, then we must also add this constant, ${\color{sthlmRed}k}$, to the right-hand expression, $x^2 +\dfrac{b}{a}x+{\color{sthlmRed}k} = \neg \dfrac{c}{a}+{\color{sthlmRed}k}$.  To determine the values of both ${\color{sthlmRed}m}$ and ${\color{sthlmRed}k}$ we should refer to the the organization of the two factors that make up the product of a quadratic expression, $x^2 +\dfrac{b}{a}x+{\color{sthlmRed}k}= ({\color{sthlmBlue}x}+{\color{sthlmRed}m})^2$.\\

%\begin{figure}[htbp]
%\begin{center}
%\begin{tikzpicture}[description/.style={fill=white, inner sep=2pt}]
%	\matrix (m) [matrix of math nodes, row sep=3em,
%	column sep=2.5em, text height=1.5ex, text depth=0.25ex]
%	{\textcolor{sthlmBlue}{x} 	&&	\textcolor{sthlmRed}{m} &	\textcolor{sthlmRed}{m}\textcolor{sthlmBlue}{x} \\
%	 \textcolor{sthlmBlue}{x}	&&	\textcolor{sthlmRed}{m}  & \textcolor{sthlmRed}{m}\textcolor{sthlmBlue}{x} \\
%	 \underbrace{\textcolor{sthlmBlue}{x \cdot x}}_{\text{1st Term}} && \underbrace{\textcolor{sthlmRed}{k}}_{\text{Last Term}} & \underbrace{\dfrac{b}{a}x}_{\text{Middle Terms}}\\};
%	\path[->, font=\scriptsize]
%	(m-1-1) 	edgenode[description] {$\times$} (m-2-3)
%			edge  node[description] {$\times$} (m-2-1)
%	(m-2-1)	edge  node[description] {$\times$} (m-1-3)
%	(m-1-3)	edge  (m-1-4)
%	(m-2-3)	edge (m-2-4)
%	(m-1-4)	edge node[description] {$+$}(m-2-4)
%	(m-2-1)	edge  (m-3-1)
%	(m-2-3)	edge  (m-3-3)
%	(m-2-4)	edge  (m-3-4)
%	(m-1-3)	edge  node[description] {$\times$} (m-2-3);
%\end{tikzpicture}

%\caption{The Organization of the Distributive Property}
%\end{center}
%\end{figure}

Since both factors of this new quadratic expression are the same, both terms that make up the middle term,  must also be the same.  We know that ${\color{sthlmRed}m}{\color{sthlmBlue}x} + {\color{sthlmRed}m}{\color{sthlmBlue}x}=\dfrac{b}{a}x$, so we should be able to determine the value of ${\color{sthlmRed}m}$ from this equation.    If we can determine the value of ${\color{sthlmRed}m}$, then we can determine the value of  ${\color{sthlmRed}k}$.

\begin{align*}
\dfrac{b}{a}x 	& = \textcolor{sthlmRed}{m}\textcolor{sthlmBlue}{x} + \textcolor{sthlmRed}{m}\textcolor{sthlmBlue}{x}\\
		& = 2mx && \text{OOA} \eqref{eq:ooa} \\
\end{align*}

Solving for $m$,

\begin{align*}
2mx & = \dfrac{b}{a}x \\
2 \cdot m \cdot x &= \dfrac{b}{a} \cdot x  && \text{JTC} \eqref{eq:jtc} \\
\dfrac{1}{2} \left[2 \cdot m \cdot x \right] &= \dfrac{1}{2} \left[\dfrac{b}{a} \cdot x \right]  && \text{SPE} \eqref{eq:spe} + \text{MI} \eqref{eq:mi1} \\
m \cdot  x &= \dfrac{b}{2a} \cdot x  && \text{OOM} \eqref{eq:oom} \\
\left[m \cdot x \right]\dfrac{1}{x}  &= \left[\dfrac{b}{2a}  \cdot x\right] \dfrac{1}{x}  && \text{SPE} \eqref{eq:spe} + \text{MI} \eqref{eq:mi1} \\
m &= \dfrac{b}{2a}  && \text{OOM} \eqref{eq:oom}
\end{align*}


%\begin{figure}[htbp]
%\begin{center}
%\begin{tikzpicture}[description/.style={fill=white, inner sep=2pt}]
%	\matrix (m) [matrix of math nodes, row sep=3em,
%	column sep=2.5em, text height=1.5ex, text depth=0.25ex]
%	{\textcolor{sthlmBlue}{x} 	&&	\textcolor{sthlmRed}{m} &	\dfrac{b}{2a} \cdot x   \\
%	 \textcolor{sthlmBlue}{x}	&&	\textcolor{sthlmRed}{m}  &  \dfrac{b}{2a} \cdot x  \\
%	 \underbrace{\textcolor{sthlmBlue}{x \cdot x}}_{\text{1st Term}} && \underbrace{\textcolor{sthlmRed}{k}}_{\text{Last Term}} & \underbrace{\dfrac{b}{a} x}_{\text{Middle Terms}}\\};
%	\path[->, font=\scriptsize]
%	(m-1-1) 	edge node[description] {$\times$} (m-2-3)
%			edge  node[description] {$\times$} (m-2-1)
%	(m-2-1)	edge  node[description] {$\times$} (m-1-3)
%	(m-1-3)	edge  (m-1-4)
%	(m-2-3)	edge (m-2-4)
%	(m-1-4)	edge node[description] {$+$}(m-2-4)
%	(m-2-1)	edge  (m-3-1)
%	(m-2-3)	edge  (m-3-3)
%	(m-2-4)	edge  (m-3-4)
%	(m-1-3)	edge  node[description] {$\times$} (m-2-3);
%\end{tikzpicture}

%\caption{We now have the two terms, $\dfrac{b}{2a}x$ that make up the sum of the middle term $\dfrac{b}{a}x$}
%\end{center}
%\end{figure}

%\newpage
%\begin{figure}[htbp]
%\begin{center}
%\begin{tikzpicture}[description/.style={fill=white, inner sep=2pt}]
%	\matrix (m) [matrix of math nodes, row sep=3em,
%	column sep=2.5em, text height=1.5ex, text depth=0.25ex]
%	{\textcolor{sthlmBlue}{x} 	&&	\textcolor{sthlmRed}{b/2a}	& x \cdot b/2a \\
%	 \textcolor{sthlmBlue}{x}	&&	\textcolor{sthlmRed}{b/2a}  & x \cdot b/2a\\
%	 \underbrace{\textcolor{sthlmBlue}{x \cdot x}}_{\text{1st Term}} && \underbrace{\textcolor{sthlmRed}{(b/2a)^2}}_{\text{Last Term}} & %\underbrace{bx/a}_{\text{Middle Terms}}\\};
%	\path[->, font=\scriptsize]
%	(m-1-1) 	edge node[description] {$\times$} (m-2-3)
%			edge  node[description] {$\times$} (m-2-1)
%	(m-2-1)	edge  node[description] {$\times$} (m-1-3)
%	(m-1-3)	edge  (m-1-4)
%	(m-2-3)	edge (m-2-4)
%	(m-1-4)	edge node[description] {$+$}(m-2-4)
%	(m-2-1)	edge  (m-3-1)
%	(m-2-3)	edge  (m-3-3)
%	(m-2-4)	edge  (m-3-4)
%	(m-1-3)	edge  node[description] {$\times$} (m-2-3);
%\end{tikzpicture}

%\caption{Doing reverse Cross-Check, we find the second term of both factors to be $\dfrac{b}{2a}$ and the  third term to be $\left( \dfrac{b}{2a} \right)^2$.}
%\end{center}
%\end{figure}

\begin{align*}
\left[x^2+\dfrac{b}{a}x \right]+\left( \dfrac{b}{2a} \right)^2 &= \left[\neg \dfrac{c}{a}\right] + \left( \dfrac{b}{2a} \right)^2 && \text{SPE} \eqref{eq:spe} \\
x^2+\dfrac{b}{a}x +\left( \dfrac{b}{2a} \right)^2 &= \neg \dfrac{c}{a}+ \left( \dfrac{b}{2a} \right)^2 && \text{APA} \eqref{eq:apa2} \\
\left( x+ \dfrac{b}{2a} \right) \left( x+ \dfrac{b}{2a} \right)& = \neg \dfrac{c}{a} + \left( \dfrac{b}{2a} \right)^2 && \text{DPF} \eqref{eq:dpf2} \\
\left( x+ \dfrac{b}{2a} \right)^2 &=\neg \dfrac{c}{a} + \left( \dfrac{b}{2a} \right)^2 && \text{PoTF} \eqref{eq:potf} \\
\left( x+ \dfrac{b}{2a} \right)^2 &=\neg \dfrac{c}{a} +  \dfrac{\left(b \right)^2}{\left(2a \right)^2}  && \text{PoQPo} \eqref{eq:poqpo1} \\
\left( x+ \dfrac{b}{2a} \right)^2 &=\neg \dfrac{c}{a} +  \dfrac{b^2}{2^2a^2} && \text{PoPrPo}\\
\left( x+ \dfrac{b}{2a} \right)^2 &=\neg \dfrac{c}{a} +  \dfrac{b^2}{4a^2} && \text{OOE}\\
\left( x+ \dfrac{b}{2a} \right)^2 &=\neg \dfrac{c}{a} \cdot {\color{sthlmRed}\dfrac{4a}{4a}} +  \dfrac{b^2}{4a^2}  && \text{MId}\\
\left( x+ \dfrac{b}{2a} \right)^2 &=\neg \dfrac{c \cdot 4 \cdot a}{a \cdot 4 \cdot a} +  \dfrac{b^2}{4a^2}  && \text{JTC}\\
\left( x+ \dfrac{b}{2a} \right)^2 &=\neg \dfrac{4 \cdot a \cdot c}{4 \cdot a \cdot a} +  \dfrac{b^2}{4a^2}  && \text{CPM}\\
\left( x+ \dfrac{b}{2a} \right)^2 &=\neg \dfrac{4 \cdot a \cdot c}{4 \cdot a^2} +  \dfrac{b^2}{4a^2}  && \text{PrCBPo}\\
\left( x+ \dfrac{b}{2a} \right)^2 &=\neg \dfrac{4 a c}{4a^2} +  \dfrac{b^2}{4a^2}  && \text{CTJ}\\
\left( x+ \dfrac{b}{2a} \right)^2 &= \dfrac{\neg 4ac+b^2}{4a^2} && \text{CD}\\
\left( x+ \dfrac{b}{2a} \right)^2 &= \dfrac{b^2+ \neg 4ac}{4a^2} && \text{CPA}\\
\displaybreak
\left[\left( x+ \dfrac{b}{2a} \right)^2\right]^{\frac{1}{2}} &=\pm \left[\dfrac{b^2+ \neg 4ac}{4a^2}\right]^{\frac{1}{2}} && \text{PoI}\\
 x+ \dfrac{b}{2a}  &=\pm \left[\dfrac{b^2+ \neg 4ac}{4a^2}\right]^{\frac{1}{2}} && \text{PoPrPo}\\
x+ \dfrac{b}{2a}  &=\pm \dfrac{\left[b^2+ \neg 4ac \right]^{\frac{1}{2}}}{\left[4a^2\right]^{\frac{1}{2}}} && \text{PoQPo}\\
x+ \dfrac{b}{2a}  &=\pm \dfrac{\left[b^2+ \neg 4ac \right]^{\frac{1}{2}}}{4^{\frac{1}{2}} a} && \text{PoPrPo}\\
x+ \dfrac{b}{2a}  &=\pm \dfrac{\left[b^2+ \neg 4ac \right]^{\frac{1}{2}}}{2a} && \text{OOE}\\
\left[x+ \dfrac{b}{2a}\right] +  \neg   \dfrac{b}{2a} &= \left[\pm\dfrac{\left[b^2+ \neg 4ac \right]^{\frac{1}{2}}}{2a}\right] + \neg \dfrac{b}{2a} && \text{AI}\\
x+ \dfrac{b}{2a} +  \neg   \dfrac{b}{2a} &= \pm\dfrac{\left[b^2+ \neg 4ac \right]^{\frac{1}{2}}}{2a} + \neg \dfrac{b}{2a} && \text{APA}\\
x+ \dfrac{b}{2a} + \neg   \dfrac{b}{2a} &=\neg   \dfrac{b}{2a} \pm\dfrac{\left[b^2+ \neg 4ac \right]^{\frac{1}{2}}}{2a} && \text{CPA}\\
x &=\neg   \dfrac{b}{2a}  \pm\dfrac{\left[b^2+ \neg 4ac \right]^{\frac{1}{2}}}{2a} && \text{OOA}\\
x &= \dfrac{\neg  b  \pm {\left[b^2+ \neg 4ac \right]^{\frac{1}{2}}}}{2a} && \text{CD}\\
x &= \dfrac{\neg  b  \pm {\left[b^2 - 4ac \right]^{\frac{1}{2}}}}{2a} && \text{DOS}\\
x &= \dfrac{\neg  b  \pm {\sqrt{b^2 - 4ac}}}{2a} && \text{ETR}\\
x &= \dfrac{-  b  \pm {\sqrt{b^2 - 4ac}}}{2a} && \text{ONeg}
\end{align*}

\end{document}

