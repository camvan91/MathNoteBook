\documentclass[20150903-160354-rs2.2-MarksMathNotebook.tex]{subfiles}

\begin{document}
%-=-=-=-=-=-=-=-=-=-=-=-=-=-=-=-=-=-=-=-=-=-=-=-=
%
%	CHAPTER
%
%-=-=-=-=-=-=-=-=-=-=-=-=-=-=-=-=-=-=-=-=-=-=-=-=

\chapter{Field Properties}

\section{Summary of Field Properties}\index{Properties!Summary of Field Properties}

\begin{table}[h]
\begin{tabular}{|l|c|c|c|c|c}
\hline
Name 								& Addition 	                & Multiplication  \\
\hline
Commutative							& $a+b=b+a$		            & $a \cdot b = b \cdot a$   \\
\hline
Associative						    & $(a+b)+c = a+(b+c)$ 		& $(a \cdot b) \cdot c = a \cdot(b \cdot c)$ \\
\hline
Distributive	                   &  $a(b+c)=ab+ac$            & $(a+b)c=ac+bc$ \\
\hline
Identity 	                       & $a + \cRed{0} = a = \cRed{0}+a$		        & $a \cdot \cRed{1} = a = \cRed{1} \cdot a$ \\
\hline
Inverse                            & $a + (-a) = 0 = (-a)+a$             & $a \cdot a^{-1}=1 = a^{-1} \cdot a$ \\
\hline
\end{tabular}
\caption{Summary of the Field Properties}\label{tab:tableoffieldproperties}
\end{table}

%-=-=-=-=-=-=-=-=-=-=-=-=-=-=-=-=-=-=-=-=-=-=-=-=
%	SECTION:
%-=-=-=-=-=-=-=-=-=-=-=-=-=-=-=-=-=-=-=-=-=-=-=-=

\section{Properties of Addition}\index{Properties of Addition}

%-=-=-= DEFINITION
\begin{definition}[Commutative Property of Addition (CPA)]\index{Property!Commutative Property of Addition}

\begin{align}
\alert{a}b &= b\alert{a} \label{eq:cpa}
\end{align}
\end{definition}

%-=-=-= DEFINITION
\begin{definition}[Associative Property of Addition (APA)]\index{Property!Associative Property of Addition}
\begin{subequations}
\begin{align}
a+b+c &= (a+b)+c \label{eq:apa1} \\
a+b+c &= a+(b+c) \label{eq:apa2}
\end{align}
\end{subequations}
\end{definition}

%-=-=-= DEFINITION
\begin{definition}[Distributive Property Factoring (DPF)]\index{Property!Distributive Property Factoring}
\begin{subequations}
\begin{align}
b\alert{a} + c\alert{a} &= (b+c)\alert{a} \label{eq:dpf1} \\
\alert{a}b + \alert{a}c &= \alert{a}(b+c) \label{eq:dpf2}
\end{align}
\end{subequations}
\end{definition}

%-=-=-= DEFINITION
\begin{definition}[Additive Identity (AId)]\index{Additive Identity}
\begin{subequations}
\begin{align}
a+\alert{0} &= a \label{eq:aid1} \\
a &= a+\alert{0} \label{eq:aid2}
\end{align}
\end{subequations}
\end{definition}

%-=-=-= DEFINITION
\begin{definition}[Additive Inverse (AI)]\index{Additive Inverse}
\begin{subequations}
\begin{align}
a + \alert{(-a)} &= 0 \label{eq:ai}
\end{align}
\end{subequations}
\end{definition}




%-=-=-=-=-=-=-=-=-=-=-=-=-=-=-=-=-=-=-=-=-=-=-=-=
%	SECTION:
%-=-=-=-=-=-=-=-=-=-=-=-=-=-=-=-=-=-=-=-=-=-=-=-=

\section{Properties of Multiplication}\index{Property!Properties of Multiplication}

%-=-=-= DEFINITION
\begin{definition}[Commutative Property of Multiplication (CPM)]\index{Property!Commutative Property of Multiplication}
\begin{align}
\alert{a} \cdot b &= b \cdot \alert{a} \label{eq:cpm}
\end{align}
\end{definition}

%-=-=-= DEFINITION
\begin{definition}[Associative Property of Multiplication (APM)]\index{Property!Associative Property of Multiplication}
\begin{subequations}
\begin{align}
a\cdot b\cdot c &= (a\cdot b)\cdot c \label{eq:apm1} \\
a\cdot b\cdot c &= a\cdot (b\cdot c) \label{eq:apm2}
\end{align}
\end{subequations}
\end{definition}

%-=-=-= DEFINITION
\begin{definition}[Distributive Property Expanding (DPE)]\index{Property!Distributive Property Factoring}
\begin{subequations}
\begin{align}
\alert{a}(b+c) &= \alert{a}b + \alert{a}c \label{eq:dpe1} \\
 (b+c)\alert{a} &= b\alert{a} + c\alert{a}  \label{eq:dpe2}
\end{align}
\end{subequations}
\end{definition}

%-=-=-= DEFINITION
\begin{definition}[Multiplicative Identity (MId)]\index{Multiplicative Identity}
\begin{subequations}
\begin{align}
\alert{1}a &= a \label{eq:mid1} \\
a &= \alert{1}a \label{eq:mid2}
\end{align}
\end{subequations}
\end{definition}

\begin{remark}
If the coefficient of a univariate monomial is the multiplicative identity \ref{eq:mid1}, 1, then it is not shown in it's canonical form.
\begin{align*}
 \cBlue{C_{k}}\cRed{x^{k}}	& = \cBlue{C_{k}} \cRed{x^{k}} \\
 							& = \cBlue{1} \cRed{x^{k}} \\
                            & = \cRed{x^{k}}
\end{align*}
\end{remark}

%-=-=-= DEFINITION
\begin{definition}[Multiplicative Inverse (MI)]\index{Multiplicative Inverse}
\begin{subequations}
\begin{align}
a \cdot \alert{\dfrac{1}{a}} &= 1 \label{eq:mi1} \\
a \cdot \alert{a^{-1}} &= 1 \label{eq:mi2}
\end{align}
\end{subequations}
\end{definition}

%-=-=-=-=-=-=-=-=-=-=-=-=-=-=-=-=-=-=-=-=-=-=-=-=
%	SECTION:
%-=-=-=-=-=-=-=-=-=-=-=-=-=-=-=-=-=-=-=-=-=-=-=-=

\section{Properties of Subtraction}\index{Property!Properties of Subtraction}



%-=-=-= DEFINITION
\begin{definition}[Definition of Subtraction (DOS)]\index{Definition of Subtraction}
\begin{subequations}
\begin{align}
a-b &= a+\neg b \label{eq:dos1} \\
a+\neg b &= a-b \label{eq:dos2}
\end{align}
\end{subequations}
\end{definition}


%-=-=-= DEFINITION
\begin{definition}[Natural Numbers]\index{Number System! Natural Numbers}

\[
\mathbb{N}=\set{0, 1, 2, 3 \ldots}
\]

\end{definition}

\begin{remark}
It is not uncommon for zero to be excluded from the natural numbers.  In fact, some exclude zero from the natural numbers and then describe the set of natural numbers that include zero the whole numbers. \\

\[
\mathbb{W}=\set{0, 1, 2, 3, \ldots}
\]

For the purposes of these notes, zero will be included within the set of natural numbers.
\end{remark}

%-=-=-= DEFINITION
\begin{definition}[Operation of Addition (OOA)]\index{Operation!Operation of Addition}
\begin{align}
\underbrace{\underbrace{a}_{\text{Augend}}+\underbrace{b}_{\text{Addend}}}_{\text{Sum}} \label{eq:ooa}
\end{align}

More generally,


\begin{align}
\underbrace{\underbrace{a}_{\text{Summand}}+\underbrace{b}_{\text{Summand}}}_{\text{Sum}} \label{eq:ooag}
\end{align}
\end{definition}

%-=-=-= DEFINITION
\begin{definition}[Operation of Multiplication (OOM)] \index{Operation!Operation of Multiplication}
\begin{align}
\underbrace{\underbrace{a}_{\text{Multiplicand}} \times \underbrace{b}_{\text{Multiplier}}}_{\text{Product}} \label{eq:oom}
\end{align}

More generally,

\begin{align}
\underbrace{\underbrace{a}_{\text{Factor}} \times \underbrace{b}_{\text{Factor}}}_{\text{Product}} \label{eq:oomg}
\end{align}

\end{definition}



\begin{definition}[Integers]\index{Number System! Integers}

\[
\mathbb{Z}=\set{\ldots, -3, -2, -1, 0, 1, 2, 3, \ldots}
\]

\end{definition}

%-=-=-= DEFINITION
\begin{definition}[Positive Integers]\index{Positive Integers}

\[
\mathbb{Z}^{+}=\set{1, 2, 3, \ldots}
\]
\end{definition}


%-=-=-= DEFINITION
\begin{definition}[Greatest Common Divisor]\index{Greatest Common Divisor}

Suppose that $m$ and $n$ are positive integers.  The greatest common divisor is the largest divisor (factor) common to both $m$ and $n$.

\end{definition}

%-=-=-= DEFINITION
\begin{definition}[Relatively Prime]\index{Relatively Prime}

Two integers $m$ and $n$ are relatively prime to each other, $m \perp n$, if they share no common positive integer divisors (factors) except 1.

\[
m \perp n \, \text{if} \, \gcd(m, n)=1.
\]

\end{definition}

\begin{definition}[Rational Numbers]\index{Number System! Rational Numbers}

\[
\mathbb{Q}=\set{m/n \suchthat m,n \in \mathbb{Z}, n \ne 0}
\]
\end{definition}

%-=-=-= DEFINITION
\begin{definition}[Proper Fraction]\index{Proper Fraction}

Given $m<n$, then the fraction $m/n$ is called \alert{proper}.

\end{definition}

%-=-=-= DEFINITION
\begin{definition}[Improper Faction]\index{Improper Faction}

Given $m>n$, then the fraction $m/n$ is called \alert{improper}.

\end{definition}

%-=-=-= DEFINITION
\begin{definition}[Common Denominator (CD)]\index{Common Denominator}
\begin{subequations}
\begin{align}
\dfrac{a}{b} + \dfrac{c}{b} &= \dfrac{a+c}{b} \label{eq:cd1} \\
\dfrac{a+c}{b}&= \dfrac{a}{b} + \dfrac{c}{b} \label{eq:cd2}
\end{align}
\end{subequations}
\end{definition}

%-=-=-= RULE
\begin{arule}[Fraction Operation of Addition (FOOA)]\index{Fraction Operation of Addition}
\begin{subequations}
\begin{align}
\dfrac{a}{b} + \dfrac{c}{d} &= \dfrac{ad+bc}{bd} \label{eq:fooa1} \\
\dfrac{ad+bc}{bd} &= \dfrac{a}{b} + \dfrac{c}{d} \label{eq:fooa2}
\end{align}
\end{subequations}
\end{arule}

\end{document}

