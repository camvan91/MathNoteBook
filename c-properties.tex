\documentclass[20150903-160354-rs2.2-MarksMathNotebook.tex]{subfiles}

\begin{document}
%-=-=-=-=-=-=-=-=-=-=-=-=-=-=-=-=-=-=-=-=-=-=-=-=
%
%	CHAPTER
%
%-=-=-=-=-=-=-=-=-=-=-=-=-=-=-=-=-=-=-=-=-=-=-=-=

\chapter{Properties}

\section{Summary of Field Properties}\index{Properties!Summary of Field Properties}

\begin{table}[h]
\begin{tabular}{|l|c|c|c|c|c}
\hline
Name 								& Addition 	                & Multiplication  \\
\hline
Commutative							& $a+b=b+a$		            & $a \cdot b = b \cdot a$   \\
\hline
Associative						    & $(a+b)+c = a+(b+c)$ 		& $(a \cdot b) \cdot c = a \cdot(b \cdot c)$ \\
\hline
Distributive	                   &  $a(b+c)=ab+ac$            & $(a+b)c=ac+bc$ \\
\hline
Identity 	                       & $a + \cRed{0} = a = \cRed{0}+a$		        & $a \cdot \cRed{1} = a = \cRed{1} \cdot a$ \\
\hline
Inverse                            & $a + (-a) = 0 = (-a)+a$             & $a \cdot a^{-1}=1 = a^{-1} \cdot a$ \\
\hline
\end{tabular}
\caption{Summary of the Field Properties}\label{tab:tableoffieldproperties}
\end{table}

%-=-=-=-=-=-=-=-=-=-=-=-=-=-=-=-=-=-=-=-=-=-=-=-=
%	SECTION:
%-=-=-=-=-=-=-=-=-=-=-=-=-=-=-=-=-=-=-=-=-=-=-=-=

\section{Properties of Addition}\index{Properties of Addition}

%-=-=-= DEFINITION
\begin{property}[Commutative Property of Addition (CPA)]\index{Property!Commutative Property of Addition}

\begin{align}
\alert{a}b &= b\alert{a} \label{eq:cpa}
\end{align}
\end{property}

%-=-=-= DEFINITION
\begin{property}[Associative Property of Addition (APA)]\index{Property!Associative Property of Addition}
\begin{subequations}
\begin{align}
a+b+c &= (a+b)+c \label{eq:apa1} \\
a+b+c &= a+(b+c) \label{eq:apa2}
\end{align}
\end{subequations}
\end{property}

%-=-=-= DEFINITION
\begin{property}[Distributive Property Factoring (DPF)]\index{Property!Distributive Property Factoring}
\begin{subequations}
\begin{align}
b\alert{a} + c\alert{a} &= (b+c)\alert{a} \label{eq:dpf1} \\
\alert{a}b + \alert{a}c &= \alert{a}(b+c) \label{eq:dpf2}
\end{align}
\end{subequations}
\end{property}

%-=-=-= DEFINITION
\begin{property}[Additive Identity (AId)]\index{Additive Identity}
\begin{subequations}
\begin{align}
a+\alert{0} &= a \label{eq:aid1} \\
a &= a+\alert{0} \label{eq:aid2}
\end{align}
\end{subequations}
\end{property}

%-=-=-= DEFINITION
\begin{property}[Additive Inverse (AI)]\index{Additive Inverse}
\begin{subequations}
\begin{align}
a + \alert{(-a)} &= 0 \label{eq:ai}
\end{align}
\end{subequations}
\end{property}


%-=-=-=-=-=-=-=-=-=-=-=-=-=-=-=-=-=-=-=-=-=-=-=-=
%	SECTION:
%-=-=-=-=-=-=-=-=-=-=-=-=-=-=-=-=-=-=-=-=-=-=-=-=

\section{Properties of Multiplication}\index{Property!Properties of Multiplication}

%-=-=-= DEFINITION
\begin{property}[Commutative Property of Multiplication (CPM)]\index{Property!Commutative Property of Multiplication}
\begin{align}
\alert{a} \cdot b &= b \cdot \alert{a} \label{eq:cpm}
\end{align}
\end{property}

%-=-=-= DEFINITION
\begin{property}[Associative Property of Multiplication (APM)]\index{Property!Associative Property of Multiplication}
\begin{subequations}
\begin{align}
a\cdot b\cdot c &= (a\cdot b)\cdot c \label{eq:apm1} \\
a\cdot b\cdot c &= a\cdot (b\cdot c) \label{eq:apm2}
\end{align}
\end{subequations}
\end{property}

%-=-=-= DEFINITION
\begin{property}[Distributive Property Expanding (DPE)]\index{Property!Distributive Property Factoring}
\begin{subequations}
\begin{align}
\alert{a}(b+c) &= \alert{a}b + \alert{a}c \label{eq:dpe1} \\
 (b+c)\alert{a} &= b\alert{a} + c\alert{a}  \label{eq:dpe2}
\end{align}
\end{subequations}
\end{property}

%-=-=-= DEFINITION
\begin{property}[Multiplicative Identity (MId)]\index{Multiplicative Identity}
\begin{subequations}
\begin{align}
\alert{1}a &= a \label{eq:mid1} \\
a &= \alert{1}a \label{eq:mid2}
\end{align}
\end{subequations}
\end{property}

\begin{remark}
If the coefficient of a univariate monomial is the multiplicative identity \ref{eq:mid1}, 1, then it is not shown in it's canonical form.
\begin{align*}
 \cBlue{C_{k}}\cRed{x^{k}}	& = \cBlue{C_{k}} \cRed{x^{k}} \\
 							& = \cBlue{1} \cRed{x^{k}} \\
                            & = \cRed{x^{k}}
\end{align*}
\end{remark}

%-=-=-= DEFINITION
\begin{property}[Multiplicative Inverse (MI)]\index{Multiplicative Inverse}
\begin{subequations}
\begin{align}
a \cdot \alert{\dfrac{1}{a}} &= 1 \label{eq:mi1} \\
a \cdot \alert{a^{-1}} &= 1 \label{eq:mi2}
\end{align}
\end{subequations}
\end{property}

%-=-=-=-=-=-=-=-=-=-=-=-=-=-=-=-=-=-=-=-=-=-=-=-=
%	SECTION:
%-=-=-=-=-=-=-=-=-=-=-=-=-=-=-=-=-=-=-=-=-=-=-=-=

\section{Properties of Subtraction}\index{Property!Properties of Subtraction}

%-=-=-= DEFINITION
\begin{definition}[Definition of Subtraction (DOS)]\index{Definition of Subtraction}
\begin{subequations}
\begin{align}
a-b &= a+\neg b \label{eq:dos1} \\
a+\neg b &= a-b \label{eq:dos2}
\end{align}
\end{subequations}
\end{definition}

%-=-=-=-=-=-=-=-=-=-=-=-=-=-=-=-=-=-=-=-=-=-=-=-=
%	SECTION:
%-=-=-=-=-=-=-=-=-=-=-=-=-=-=-=-=-=-=-=-=-=-=-=-=
\section{Properties of Powers}\index{Property!Properties of Powers}


%-=-=-= DEFINITION
\begin{property}[Power Inverse (PoId)]\index{Power Identity}
\begin{subequations}
\begin{align}
1&= b^0  \label{eq:poid1} \\
b^{0}&= 1  \label{eq:poid2}
\end{align}
\end{subequations}
\end{property}


%-=-=-=-=-=-=-=-=-=-=-=-=-=-=-=-=-=-=-=-=-=-=-=-=
%	SECTION:
%-=-=-=-=-=-=-=-=-=-=-=-=-=-=-=-=-=-=-=-=-=-=-=-=
\section{Properties of Equality}\index{Property!Properties of Equality}

%-=-=-= PROPERTY
\begin{property}[Reflexive Property of Equality (RPE)]\index{Reflexive Property of Equality}
\begin{subequations}
\begin{align}
a &= a \label{eq:rpe}
\end{align}
\end{subequations}
\end{property}

%-=-=-= PROPERTY
\begin{property}[Substitution Property of Equality (SPE)]\index{Substitution Property of Equality} \hfill \\

Given $a=b$, then
\begin{align}
E(a) &= E(b) \label{eq:spe}
\end{align}
$E(x)$ represents any expression.
\end{property}

%-=-=-= PROPERTY
\begin{property}[Symmetric Property of Equality (SyPE)]\index{Symmetric Property of Equality}
\begin{subequations}
\begin{align}
a =b \quad \text{then} \quad b=a \label{eq:sype}
\end{align}
\end{subequations}
\end{property}

%-=-=-= PROPERTY
\begin{property}[Transitive Property of Equality (TPE)]\index{Transitive Property of Equality}
\begin{subequations}
\begin{align}
\text{if} \quad a=b \quad \text{and} \quad b=c \quad \text{then} \quad a =c \label{eq:tpe}
\end{align}
\end{subequations}
\end{property}

%-=-=-= PROPERTY
\begin{property}[Zero Factor Property (ZFP)]\index{Zero Factor Property}
\begin{subequations}
\begin{align}
\text{if} \quad a \cdot b=0 \quad \text{then} \quad a=0 \quad \text{or} \quad b=0 \label{eq:zfp}
\end{align}
\end{subequations}
\end{property}

%-=-=-=-=-=-=-=-=-=-=-=-=-=-=-=-=-=-=-=-=-=-=-=-=
%	SECTION:
%-=-=-=-=-=-=-=-=-=-=-=-=-=-=-=-=-=-=-=-=-=-=-=-=
\section{Properties of Inequality}\index{Property!Properties of Inequality}

%-=-=-= PROPERTY
\begin{property}[Substitution Property of Inequality (SPIn)]\index{Substitution Property of Inequality}
\begin{subequations}
\begin{align}
a<b \quad \text{then} \quad a + c < b + c \label{eq:spin1} \\
a<b \quad \text{and} c>0 ,\quad\text{then} \quad ca<cb \label{eq:spin2} \\
a<b \quad \text{and} c<0 ,\quad\text{then} \quad ca>cb \label{eq:spin3} 
\end{align}
\end{subequations}
\end{property}

%-=-=-= PROPERTY
\begin{property}[Transitive Property of Inequality (TPIn)]\index{Transitive Property of Inequality}
\begin{subequations}
\begin{align}
\text{if} \quad a<b \quad \text{and} \quad b<c \quad \text{then} \quad a>c \label{eq:tpin}
\end{align}
\end{subequations}
\end{property}

\end{document}

