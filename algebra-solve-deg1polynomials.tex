\documentclass[20150903-160354-rs2.2-MarksMathNotebook.tex]{subfiles}

\begin{document}
%-=-=-=-=-=-=-=-=-=-=-=-=-=-=-=-=-=-=-=-=-=-=-=-=
%
%	CHAPTER
%
%-=-=-=-=-=-=-=-=-=-=-=-=-=-=-=-=-=-=-=-=-=-=-=-=

\chapter{Solving Linear Equations}

%-=-=-=-=-=-=-=-=-=-=-=-=-=-=-=-=-=-=-=-=-=-=-=-=
%	SECTION:
%-=-=-=-=-=-=-=-=-=-=-=-=-=-=-=-=-=-=-=-=-=-=-=-=
\section{Power Inverse}\index{Solving Quadratic Equations! Power Inverse}

%-=-=-= EXAMPLE
\begin{example}[id:20141206-102142] \label{20141206-102142}\index{Example!20141206-102142} \hfill \\

Solve the equation $x+a=b$ for $x$

\soln

\solnsteps
\begin{align*}
\farg{x+a} + \neg a  &= \farg{b} + \neg a  && \text{SPE} \eqref{eq:spe} + \text{AI} \eqref{eq:ai} \\
x + (a+\neg a) &= b + \neg a  && \text{APA} \eqref{eq:apa2} \\
x + 0 &= b + \neg a  && \text{OOA} \eqref{eq:ooa} \\
x &= b + \neg a  && \text{AId} \eqref{eq:aid2} \\
x &= b-a  && \text{DOS} \eqref{eq:dos2}
\end{align*}
\end{example}


%-=-=-= EXAMPLE
\begin{example}[id:20141111-222931] \label{20141111-222931}\index{Example!20141111-222931} \hfill \\

Solve the equations $x+8=0$

\soln

\solnsteps
\begin{align*}
\farg{x+8}+\neg 8 &= \farg{0}+\neg 8 && \text{SPE} \eqref{eq:spe} + \text{AI} \eqref{eq:ai} \\
x+(8+\neg 8) &= 0 + \neg 8 && \text{APA} \eqref{eq:apa2} \\
x+0 &= \neg 8 && \text{OOA} \eqref{eq:ooa} \\
x &= \neg 8 && \text{AId} \eqref{eq:aid2} \\
x &= -8 && \text{ONeg} \eqref{eq:oneg2}
\end{align*}

\soln

\lesssteps
\begin{align*}
\farg{x+8}+\neg 8 &= \farg{0}+\neg 8 && \text{SPE} \eqref{eq:spe} + \text{AI} \eqref{eq:ai} \\
x &= - 8 && \text{OOA} \eqref{eq:ooa}
\end{align*}

\qdepend

\qdependlist

example \ref{20141111-190212}-20141111-190212

\end{example}

%-=-=-= EXAMPLE
\begin{example}[id:20141206-101632] \label{20141206-101632}\index{Example!20141206-101632} \hfill \\

Solve the equation $x+4=7$

\soln

\solnsteps
\begin{align*}
\farg{x +4} + \neg 4  &= \farg{7} + \neg 4  && \text{SPE} \eqref{eq:spe} + \text{AI} \eqref{eq:ai} \\
x + (4+\neg 4) &= 7 + \neg 4  && \text{APA} \eqref{eq:apa2} \\
x + 0 &= 3  && \text{OOA} \eqref{eq:ooa} \\
x  &= 3  && \text{AId} \eqref{eq:aid2}
\end{align*}

\soln

\lesssteps

\begin{align*}
\farg{x} +4 + \neg 4  &= \farg{7} + \neg 4  && \text{SPE} \eqref{eq:spe} + \text{AI} \eqref{eq:ai} \\
x &= 3  && \text{OOA} \eqref{eq:ooa}
\end{align*}
\end{example}

%-=-=-= EXAMPLE
\begin{example}[id:20141206-101107] \label{20141206-101107}\index{Example!20141206-101107} \hfill \\

Solve the equation $x-8=15$ for $x$

\soln

\solnsteps
\begin{align*}
x+ \neg 8 &= 15  && \text{DOS} \eqref{eq:dos1} \\
\farg{x+ \neg 8} + 8  &= \farg{15} + 8  && \text{SPE} \eqref{eq:spe} + \text{AI} \eqref{eq:ai} \\
x + (\neg 8+ 8) &= 15 + 8  && \text{APA} \eqref{eq:apa2} \\
x + 0 &= 23  && \text{OOA} \eqref{eq:ooa} \\
x  &= 23  && \text{AId} \eqref{eq:aid2}
\end{align*}

\soln

\lesssteps
\begin{align*}
\farg{x+ \neg 8} + 8  &= \farg{15} + 8  && \text{SPE} \eqref{eq:spe} + \text{AI} \eqref{eq:ai} \\
x &= 23  && \text{OOA} \eqref{eq:ooa} \\
\end{align*}
\end{example}

%-=-=-= EXAMPLE
\begin{example}[id:20141206-102404] \label{20141206-102404}\index{Example!20141206-102404} \hfill \\

Solve the equation $5x=9$ for $x$.

\soln

\solnsteps
\begin{align*}
\dfrac{1}{5}\farg{5x} &= \dfrac{1}{5}\farg{9} && \text{SPE} \eqref{eq:spe} + \text{MI} \eqref{eq:mi1} \\
\dfrac{1}{5} \cdot \left[ 5 \cdot x \right] &= \dfrac{1}{5} \cdot 9 && \text{JTC} \eqref{eq:jtc} \\
\left(\dfrac{1}{5} \cdot 5 \right) \cdot x &= \dfrac{1}{5} \cdot 9  && \text{APM} \eqref{eq:apm2} \\
1\cdot x &= \dfrac{9}{5}  && \text{OOM} \eqref{eq:oom} \\
x &= \dfrac{9}{5}  && \text{MId} \eqref{eq:mid2}
\end{align*}

\soln

\lesssteps
\begin{align*}
\dfrac{1}{5}\farg{5x} &= \dfrac{1}{5}\farg{9} && \text{SPE} \eqref{eq:spe} + \text{MI} \eqref{eq:mi1} \\
x &= \dfrac{9}{5}  && \text{OOM} \eqref{eq:oom}
\end{align*}
\end{example}

%-=-=-= EXAMPLE
\begin{example}[id:20141206-104404] \label{20141206-104404}\index{Example!20141206-104404} \hfill \\

Solve the equation $ax=b$ for $x$.

\soln

\solnsteps
\begin{align*}
\dfrac{1}{a}\farg{ax} &= \dfrac{1}{a}\farg{b} && \text{SPE} \eqref{eq:spe} + \text{MI} \eqref{eq:mi1} \\
\dfrac{1}{a} \cdot \left( a \cdot x \right) &= \dfrac{1}{a} \cdot b && \text{JTC} \eqref{eq:jtc} \\
\left(\dfrac{1}{a} \cdot a \right) \cdot x &= \dfrac{1}{a} \cdot b  && \text{APM} \eqref{eq:apm2} \\
1\cdot x &= \dfrac{b}{a}  && \text{OOM} \eqref{eq:oom} \\
x &= \dfrac{b}{a}  && \text{MId} \eqref{eq:mid2}
\end{align*}
\end{example}

%-=-=-= EXAMPLE
\begin{example}[id:20141206-102723] \label{20141206-102723}\index{Example!20141206-102723} \hfill \\

Solve the equation $-2x=7$ for $x$

\soln

\solnsteps
\begin{align*}
\neg 2x &= 7  && \text{ONeg} \eqref{eq:oneg1} \\
\neg \dfrac{1}{2}\farg{\neg 2x}  &= \neg \dfrac{1}{2}\farg{7} && \text{SPE} \eqref{eq:spe} + \text{MI} \eqref{eq:mi2} \\
\neg \dfrac{1}{2} \cdot \left( \neg 2 \cdot x \right) &= \neg \dfrac{1}{2} \cdot 7 && \text{JTC} \eqref{eq:jtc} \\
\left(\neg\dfrac{1}{2} \cdot \neg 2 \right) \cdot x &= \neg \dfrac{1}{2} \cdot 7  && \text{APM} \eqref{eq:apm2} \\
1\cdot x &= \neg \dfrac{7}{2}  && \text{OOM} \eqref{eq:oom} \\
1\cdot x &= -\dfrac{7}{2}  && \text{ONeg} \eqref{eq:oneg2} \\
x &= -\dfrac{7}{2}  && \text{MId} \eqref{eq:mid2}
\end{align*}

\soln

\lesssteps
\begin{align*}
\neg \dfrac{1}{2}\farg{\neg 2x}  &= \neg \dfrac{1}{2}\farg{7} && \text{SPE} \eqref{eq:spe} + \text{MI} \eqref{eq:mi2} \\
x &= -\dfrac{7}{2}  && \text{OOM} \eqref{eq:oom}
\end{align*}
\end{example}


%-=-=-= EXAMPLE
\begin{example}[id:20141111-215726] \label{20141111-215726}\index{Example!20141111-215726} \hfill \\

Solve the equation $2x+5=0$ for $x$

\soln

\solnsteps
\begin{align*}
\farg{2x+5}+\neg 5 &= \farg{0} + \neg 5 && \text{SPE} \eqref{eq:spe} + \text{AI} \eqref{eq:ai} \\
2x+(5+\neg 5) &= 0+\neg 5 && \text{APA} \eqref{eq:apa1} \\
2x + 0 &= \neg 5 && \text{OOA} \eqref{eq:ooa} \\
2x &= \neg 5 && \text{AId} \eqref{eq:aid1} \\
\dfrac{1}{2}\farg{2x} &= \dfrac{1}{2}\farg{\neg 5} && \text{SPE} \eqref{eq:spe} + \text{MI} \eqref{eq:mi1} \\
\dfrac{1}{2} \cdot 2 \cdot x &= \dfrac{1}{2} \cdot \neg 5 && \text{JTC} \eqref{eq:jtc} \\
\left(\dfrac{1}{2}\cdot 2 \right) \cdot x &= \dfrac{1}{2} \cdot \neg 5 && \text{APM} \eqref{eq:apm1} \\
1 \cdot x &= \dfrac{\neg 5}{2} && \text{OOM} \eqref{eq:oom} \\
1x &= -\dfrac{5}{2} && \text{ONeg} \eqref{eq:oneg2} \\
x &= -\dfrac{5}{2} && \text{MId} \eqref{eq:mid2}
\end{align*}

\soln

\lesssteps
\begin{align}
\farg{2x+5}+\neg 5 &= \farg{0} + \neg 5 && \text{SPE} \eqref{eq:spe} + \text{AI} \eqref{eq:ai} \\
2x &= \neg 5 && \text{OOA} \eqref{eq:ooa} \\
\dfrac{1}{2}\farg{2x} &= \dfrac{1}{2}\farg{\neg 5} && \text{SPE} \eqref{eq:spe} + \text{MI} \eqref{eq:mi1} \\
x &= -\dfrac{5}{2} && \text{OOM} \eqref{eq:oom}
\end{align}

\qdepend

\qdependlist

example \ref{20141111-192213}-20141111-192213


\end{example}

\end{document}

